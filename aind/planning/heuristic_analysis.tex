\documentclass[10pt,a4paper]{article}
\usepackage[utf8]{inputenc}
\usepackage{amsmath}
\usepackage{amsfonts}
\usepackage{amssymb}
\usepackage{graphicx}

\begin{document}

\title{Part 3: Written Analysis}
\author{Mario Namtao Shianti Larcher}

\maketitle

\begin{abstract}

In this report I analyze different heuristic and non-heuristic search methods in the context of deterministic logistics planning problems.

\end{abstract}

\section{Optimal plan for Problems 1, 2, and 3}

In table \ref{tab:opt} are reported three possible optimal plans for problems 1, 2 and 3. These solutions are the result of \textit{bread-first search} that is optimal if the path cost is a non-decreasing function of the depth of the node, as in our case.

\begin{table}
\resizebox{\linewidth}{!}{%
\begin{tabular}{|c|c|c|c|}
\hline 
Action \# & Problem 1 & Problem 2 & Problem 3 \\ 
\hline 
1 & $Load(C1, P1, SFO)$ & $Load(C1, P1, SFO)$ & $Load(C1, P1, SFO)$ \\ 
\hline 
2 & $Load(C2, P2, JFK)$ & $Load(C2, P2, JFK)$ & $Load(C2, P2, JFK)$ \\ 
\hline 
3 & $Fly(P2, JFK, SFO)$  & $Load(C3, P3, ATL)$ & $Fly(P2, JFK, ORD)$ \\ 
\hline 
4 & $Unload(C2, P2, SFO)$ & $Fly(P2, JFK, SFO)$ & $Load(C4, P2, ORD)$ \\ 
\hline 
5 & $Fly(P1, SFO, JFK)$ & $Unload(C2, P2, SFO)$ & $Fly(P1, SFO, ATL)$ \\ 
\hline 
6 & $Unload(C1, P1, JFK)$ & $Fly(P1, SFO, JFK)$ & $Load(C3, P1, ATL)$ \\ 
\hline 
7 & • & $Unload(C1, P1, JFK)$  & $Fly(P1, ATL, JFK)$ \\ 
\hline 
8 & • & $Fly(P3, ATL, SFO)$ & $Unload(C1, P1, JFK)$ \\ 
\hline 
9 & • & $Unload(C3, P3, SFO)$ & $Unload(C3, P1, JFK)$ \\ 
\hline 
10 & • & • & $Fly(P2, ORD, SFO)$ \\ 
\hline 
11 & • & • & $Unload(C2, P2, SFO)$ \\ 
\hline 
12 & • & • & $Unload(C4, P2, SFO)$ \\ 
\hline 
\end{tabular}%
}
\caption{Optimal plan for Problems 1, 2, and 3.}\label{tab:opt}
\end{table}

\section{Comparison and contrast of non-heuristic search result metrics}

\begin{table}
\resizebox{\linewidth}{!}{%
\begin{tabular}{|c|c|c|c|c|c|c|}
\hline 
Problem \# & Search Algorithm & Expansions & Goal Tests & New Node & Plan Length & Time Elapsed (s) \\ 
\hline 
1 & Breadth-first search & 43 & 56 & 180 & 6 & 0.03 \\ 
\hline 
1 & Depth-first search & 21 & 22 & 84 & 20 & 0.01 \\ 
\hline 
1 & Uniform-cost search & 55 & 57 & 224 & 6 & 0.03 \\ 
\hline 
2 & Breadth-first search & 3343 & 4609 & 30509 & 9 & 8.21 \\ 
\hline 
2 & Depth-first search & 624 & 625 & 5602 & 619 & 3.36 \\ 
\hline 
2 & Uniform-cost search & 4852 & 4854 & 44030 & 9 & 10.92 \\ 
\hline 
3 & Breadth-first search & 14663 & 18098 & 129631 & 12 & 40.37 \\ 
\hline 
3 & Depth-first search & 408 & 409 & 3364 & 392 & 1.63 \\ 
\hline 
3 & Uniform-cost search & 18235 & 18237 & 159716 & 12 & 48.16 \\ 
\hline
\end{tabular}%
}
\caption{Comparison and contrast of non-heuristic search result metrics.}\label{tab:nonheu}
\end{table}

In table \ref{tab:nonheu} is presented a comparison between three uninformed search strategies: \textit{breadth-first search}, \textit{depth-first search} and \textit{uniform-cost search}. From the results we can notice some expected behaviors, \textit{breadth-first search} and  \textit{uniform-cost search} are always optimal (as already observed our path cost is a non-decreasing function of the depth of the node), \textit{uniform-cost search} is strictly less efficient than \textit{breadth-first search} because our step costs are equal and this technique examines all the nodes at the goal's depth to see if one has a lower cost where \textit{breadth-first search} as soon as it generates a goal. Finally  \textit{depth-first search} is not optimal (the solution are in fact significantly longer that the optimal length) but is faster in all three problems, this is not always guaranteed and depends on the problem.

\section{Comparison and contrast of heuristic search result metrics}

\begin{table}
\resizebox{\linewidth}{!}{%
\begin{tabular}{|c|c|c|c|c|c|c|}
\hline 
Problem \# & Search Algorithm & Expansions & Goal Tests & New Node & Plan Length & Time Elapsed (s) \\ 
\hline 
1 & $A^{*}$ - \emph{ignore preconditions} & 41 & 43 & 170 & 6 & 0.03 \\ 
\hline 
1 & $A^{*}$ - \emph{level-sum} & 11 & 13 & 50 & 6 & 0.76 \\ 
\hline 
2 & $A^{*}$ - \emph{ignore preconditions} & 1450 & 1452 & 13303 & 9 & 3.83 \\ 
\hline 
2 & $A^{*}$ - \emph{level-sum} & 86 & 88 & 841 & 9 & 69.90 \\ 
\hline 
3 & $A^{*}$ - \emph{ignore preconditions} & 5040 & 5042 & 44944 & 12 & 15.45 \\ 
\hline 
3 & $A^{*}$ - \emph{level-sum} & 325 & 327 & 3002 & 12 & 360.31 \\ 
\hline 
\end{tabular}%
}
\caption{Comparison and contrast of heuristic search result metrics.}\label{tab:heu}
\end{table}

The graph version of $A^{*}$ search is optimal if the heuristic used is consistent, \textit{level-sum} its not consistent and then we don't have 
guarantees of optimality. \emph{ignore preconditions} is consistent so we know that $A^{*}$ - \emph{ignore preconditions} is optimal. For our three problem both the heuristics (with $A^{*}$) find optimal solutions, \emph{level-sum} do this expanding less node where \emph{ignore preconditions} use less time to reach a goal state (this can be due also by an inefficient implementation of \emph{level-sum}).

\section{Conclusions}

For the problems under analysis $A^{*}$ - \emph{ignore preconditions} is  strictly better of all optimal search methods in term of time and expansions. The fastest search in all the problems is \emph{depth-first search} but the solutions found by this method are significantly longer than optimal solutions.

\end{document}